\documentclass[main.tex]{subfiles}

\begin{document}
\newpage
\begin{boxnaslovi}
\section{Pitanja}
\end{boxnaslovi}

\begin{multicols}{2}

\subsection{Uvod}
\begin{enumerate}

\item Značenje reči paradigma i programska paradigma.
\item Uloga programskih paradigmi.
\item Definicija programskog jezika.
\item Povezanost paradigmi i jezika.
\item Razvoj programskih jezika.

\end{enumerate}

\subsection{Osnovne paradigme}
\begin{enumerate}
\item Šta karakteriše proceduralnu paradigmu?
\item Šta karakteriše deklarativnu paradigmu?
\item Koje su osnovne četiri programske paradigme?
\item Nabroj bar četiri dodatne programske paradigme.
\item Koje su osnovne karakteristike imperativne paradigme?
\item Nabroj tri jezika koji pripadaju imperativnoj paradigmi.
\item Koje su osnovne karakteristike ... paradigme?
\item Nabroj tri jezika koji pripadaju ... paradigmi.
\item Šta je programska paradigma?
\item Koje su osnovne programske paradigme?
\item Šta je programski jezik?
\item Koji je odnos programskih jezika i programskih paradigmi?
\item Zašto su nastajale i nastaju nove programske paradigme?

\end{enumerate}

\subsection{Dodatne paradigme}
\begin{enumerate}
\item Koje su osnovne karakteristike komponentne paradigme?
\item Nabroj tri jezika koji pripadaju (podržavaju) komponentnu paradigmu.
\item Koje su osnovne karakteristike ... paradigme?
\item Nabroj tri jezika koji pripadaju ... paradigmi.

\end{enumerate}

\subsection{Funkcionalna paradigma}
\begin{enumerate}
\item Na koji način je John Backus uticao na razvoj funkcionalnih jezika?
\item Koji su najpoznatiji funkcionalni programski jezici?
\item Koji je domen upotrebe funkcionalnih programskih jezika?
\item Koje su osnovne karakteristike funkcionalnih programskih jezika?
\item Šta je svojstvo referentne prozirnosti i na koji način ovo svojstvo utiče na redosled naredbi u funkciji?
\item Koje su osobine programa u kojima se poštuje pravilo referentne prozirnosti?
\item Da li je moguće u potpunosti zadržati svojstvo referentne prozirnosti?
\item Koji je odnos referentne prozirnosti sa bočnim efektima?
\item Da li je moguće obezbediti promenu stanja programa i istovremeno zadržati svojstvo referentne prozirnosti?
\item Šta su funkcionalni jezici? Šta su čisto funkcionalni jezici?
\item Navesti primere čisto funkcionalnih jezika?
\item Koje su osnovne aktivnosti u okviru funkcionalnog programiranja?
\item Kako izgleda program napisan u funkcionalnom programskom jeziku?
\item Šta je potrebno da obezbedi funkcionalni programski jezik za uspešno proramiranje?
\item Šta je striktna/nestriktna semantika?
\item Kakvu semantiku ima jezik Haskell?
\item Kakvu semantiku ima jezik Lisp?
\item Koje su prednosti funkcionalnog programiranja?
\item Koje su mane funkcionalnog programiranja?
\item Šta uključuje definisanje funkcije?
\item Šta su funkcije višeg reda? Navesti primere.
\item Da li matematičke funkcije imaju propratne efekte?
\end{enumerate}

\subsection{Lambda racun i Haskell}
\begin{enumerate}
\item Koji je formalni okvir funkcionalnog programiranja?
\item Koji se jezik smatra prvim funkcionalnim jezikom?
\item Koja je ekspresivnost lambda računa?
\item Koji su sve sinonimi za lambda izraz?
\item Navesti definiciju lambda terma.
\item Da li čist lambda račun uključuje konstante u definiciji?
\item Navesti primer jednog lambda izraza, objasniti njegovo značenje i primeniti dati izraz na neku konkretnu vrednost.
\item Koja je asocijativnost primene a koja apstrakcije?
\item Navesti ekvivalentan izraz sa zagradama za izraz ...
\item Koje su slobodne a koje vezane promenljive u izrazu ...
\item Navesti definiciju slobodne promenljive? Koje promenljive su vezane?
\item Koja je uloga alfa ekvivalentnosti?
\item Šta su redukcije?
\item Šta je delta redukcija? Navesti primer.
\item Šta je alfa redukcija? Navesti primer.
\item Kada se koristi alfa redukcija?
\item Šta je beta redukcija? Navesti primer.
\item Definisati supstituciju.
\item Navesti primer lambda izraza koji definiše funkciju višeg reda koja prima funkciju kao argument.
\item Navesti primer lambda izraza koji definiše funkciju višeg reda koja ima funkciju kao povratnu vrednost.
\item Čemu služi Karije postupak?
\item Kako se definišu funkcije sa više argumenata?
\item Šta je normalni oblik funkcije?
\item Da li svi izrazi imaju svoj normalni oblik?
\item Navesti svojstvo konfluentnosti.
\item Da li izraz može imati više normalnih obilika?
\item Koja je razlika izmedu aplikativnog i normalnog poretka?
\item Šta govori teorema standardizacije?
\item Šta se dobija lenjom evaluacijom?
\item Koje su osnovne karakteristike Haskela?
\item Šta izračunava naredni Haskel program ...

\end{enumerate}

\subsection{Konkurentno programiranje}
\begin{enumerate}
\item Šta je konkurentna paradigma?
\item Da li su ideje o konkurentnosti nove? Zbog čega je konkurentnost važna?
\item Koji su osnovni nivoi konkurentnosti?
\item Koje su vrste konkurentnosti u odnosu na hardver? Kako se to odnosi na programere i dizajn programskog jezika?
\item Šta je osnovni cilj koji želimo da ostvarimo razvijanjem konkurentnih algoritama?
\item Koji su osnovni razlozi za korišćenje konkurentnog programiranja?
\item Navesti primer upotrebe konkurentnog programiranja za podršku logičkoj strukturi programa.
\item Da li je dobijanje na brzini moguće ostvariti i na jednoprocesorskoj mašini?
\item Koji je hijerarhijski odnos u okviru konkurentne paradigme?
\item Šta je zadatak? Na koji način se zadatak razlikuje od potprograma?
\item Koje su osnovne kategorije zadataka i koje su karakteristike ovih kategorija?
\item Šta je paralelizacija zadataka?
\item Šta je paralelizacija podataka?
\item Navesti primere paralelizacije zadataka i paralelizacije podataka.
\item Koji je odnos ovih paralelizacija?
\item Šta je komunikacija?
\item Koji su osnovni mehanizmi komunikacije?
\item Šta karakteriše slanje poruka u okviru iste mašine, a šta ukoliko je slanje poruka preko mreže?
\item Šta je sinhronizacija?
\item Kakva je sinhronizacija u okviru modela slanja poruka?
\item Kakva je sinhronizacija u okviru modela deljene memorije?
\item Koje su dve osnovne vrste sinhronizacije u okviru modela deljene memorije?
\item Objasniti sinhronizaciju saradnje.
\item Šta je uslov takmičenja?
\item Koji su načini implementiranja sinhronizacije?
\item Šta je koncept napredovanja?
\item Navesti primer uzajamnog blokiranja.
\item Navesti primer živog blokiranja.
\item Navesti primer individualnog izgladnjivanja.
\item Koje vrste uzajamnog isključivanja postoje?
\item Koji je odnos konkurentnosti i potprograma/klasa.
\item Opisati problem filozofa za večerom.
\item Semantika muteksa i katanaca.
\end{enumerate}

\subsection{Logičko programiranje i Prolog}
\begin{enumerate}
\item Šta čini teorijske osnove logičkog programiranja?
\item Na koji način se rešavaju problemi u okviru logičke paradigme?
\item Koji su osnovni predstavnici logičke paradigme?
\item Za koju vrstu problema je pogodno koristiti logičko programiranje?
\item Za koju vrstu problema nije pogodno koristiti logičko programiranje?
\item Definisati logičke i nelogičke simbole logike prvog reda.
\item Definisati term logike prvog reda.
\item Definisati atomičku formulu logike prvog reda.
\item Šta je literal? Šta je klauza?
\item Definisati supstituciju za termove.
\item Definisati supstituciju za atomičke formule.
\item Ukoliko je zadata supstitucija $\sigma = \ldots$ i term $t = \ldots$ izračunati $t\sigma$
\item Šta je problem unifikacije?
\item Kada kažemo da su izrazi unifikabilni?
\item Da li za dva izraza uvek postoji unifikator?
\item Ukoliko za dva izraza postoji unifikator, da li on mora da bude jedinstven?
\item Ukoliko su dati termovi $t?1 = \ldots$ i $t_2 = \ldots$ izračunati jedan unifikator ovih termova.
\item Šta je metod rezolucije?
\item Šta čini teorijske osnove logičkog programiranja?
\item Na koji način se rešavaju problemi u okviru logičke paradigme?
\item Koji su osnovni predstavnici logičke paradigme?
\item Za koju vrstu problema je pogodno koristiti logičko programiranje?
\item Za koju vrstu problema nije pogodno koristiti logičko programiranje?
\item Šta je Hornova klauza i čemu ona odgovara?
\item Šta je supstitucija?
\item Kada se dva terma mogu unifikovati?
\item Od čega se sastoji programiranje u Prologu?
\item Šta su činjenice, šta se pomoću njih opisuje?
\item Šta su pravila i šta se pomoću njih zadaje?
\item Šta odreduju činjenice i pravila?
\item Šta govori pretpostavka o zatvorenosti?
\item Šta su upiti i čemu oni služe?
\item Šta su termovi?
\item Kako se vrši unifikacija nad termima u Prologu?
\item Šta je lista?
\item Napisati pun i skraćen zapis liste od tri elementa a, b i c.
\item Šta omogućava metaprogramiranje?
\item Napisati deklarativno tumačenje naredne Prolog konstrukcije ...
\item Napisati proceduralno tumačenje naredne Prolog konstrukcije ...
\item Šta je stablo izvodenja i čemu ono odgovara u smislu deklarativne/proceduralne
semantike?
\item Koji su osnovni elementi stabla izvodenja?
\item Nacrtati stablo izvodenja za naredni Prolog program ...
\item Na koji način redosled tvrdenja u bazi znanja utiče na pronalaženje rešenja
u Prologu?
\item Koja je uloga operatora sečenja?
\item Nacrtati stablo izvodenja za naredni program bez operatora sečenja, i
sličan program sa operatorom sečenja ...
\item Šta je crveni a šta zeleni operator sečenja?
\item Koja je uloga opreatora sečenja u narednom primeru ...
\item Da li se Hornovim klauzama mogu opisati sva tvrdenja logike prvog reda?
\item Šta Prolog ne može da dokaže?
\item Koje su osobine NOT operatora?
\item Da li Prolog može da obezbedi generisanje efikasnih algoritama?
\item Kakva je kompatibilnost izmedu različitih Prolog kompajlera?
\item Da li je Prolog Tjuring kompletan jezik?
\item Kakav je sistem tipova u prologu?
\item Šta je programiranje ograničenja?
\item Koji su predstavnici paradigme programiranja ograničenja?
\item Po čemu se razlikuje izraz $ x < y$ u imperativnoj paradigmi i paradigmi
ograničenja?
\item Od čega se sastoji programiranje ograničenja nad konačnim domenom?
\item Napisati program u B-Prologu koji pronalazi sve vrednosti promenljivih
X, Y i Z za koje važi da je $X \le Y$ i $X + Y \le Z$ pri čemu promenljive
pripadaju narednim domenima $X \in \{1, 2, ..., 50\}$, $Y \in \{5, 10, ..100\}$ i $Z \in
\{1, 3, 5, ...99\}$
\end{enumerate}

\subsection{Imperativna paradigma}
\begin{enumerate}
\item Pod kakvim uticajem je nastala imperativna paradigma?
\item Šta je stanje programa?
\item Koje su faze razvoja imperativne paradigme?
\item Koje su karakteristike operacione paradigme?
\item Koji je minimalni skup naredbi operaciona pardigme?
\item Koje su karakteristike strukturne paradigme?
\item Koji je minimalni skup naredbi strukturne pardigme?
\item Koje su karakteristike proceduralne paradigme?
\item Koje vrste prenosa parametara postoje?
\item Kako je u memoriji orgranizovano izvršavanje potprograma?
\item Šta su korutine?
\item Koje su karakteristike modularne paradigme?
\item Šta omogućava modularna paradigma?
\item Kako se rešavaju problemi u okviru proceduralne paradigme?
\item Šta su bočni efekti?
\item Do čega dovode bočni efekti?
\end{enumerate}

\subsection{Objektno-orijentisano programiranje}
\begin{enumerate}
\item  Koji su principi funkcionalne dekompozicije i koji su osnovni problemi ovoga pristupa?
\item Šta je osnovni uzrok problema kod rešavanja funkcionalnom dekompozicijom?
\item Zašto je uticaj izmena zahteva važan?
\item Šta je kohezija, a šta kopčanje i kako su povezani?
\item Šta je efekat talasanja i da li je on poželjan?
\item Koji su bili simptomi prve softverske krize?
\item Šta je apstrakcija?
\item Šta je interfejs?
\item Šta implementacija?
\item Objasniti odnos interfejsa i implementacije.
\item Šta je enkapsulacija?
\item Koji je odnos apstrakcije i enkapsulacije?
\item Šta je objekat? (filozofski? konceptulano? u objektnoj terminologiji? specifikacijski? implementaciono?)
\item Kako komuniciraju objekti?
\item Šta je klasa?
\item Koji je odnos klase i objekta?
\item Koji je prvi objektni jezik i kada je nastao?
\item Šta su objektno zasnovani, a šta objektno orijenitsani jezici?
\item Koji su najpopularniji objektno orijentisani jezici?
\item Šta je nasleđivanje?
\item Na koje načine se koristi nasledivanje? Šta je proširivanje, a šta specijalizacija?
\item Šta omogućava nasledivanje?
\item Šta je višestruko nasledivanje?
\item Koji jezici omogućavaju višestruko nasleđivanje, a koji ne?
\item Koje su osnovne vidljivosti koje klase definišu?
\item Šta je polimorfizam?
\item Koja je razlika izmedu preopterećivanja i predefinisanja?
\item Kada se koristi statičko a kada dinamičko vezivanje?
\item Šta definišu apstraktne klase?
\item Koje su mogućnosti generičkog programiranja?
\item Obrazložiti sličnosti i razlike strukturnog i OO programiranja?
\item Koje su osnovne prednosti OO programiranja u odnosu na strukturno programiranje?
\end{enumerate}

\subsection{Osnovna svojstva programskih jezika}
\begin{enumerate}
\item Koja su osnovna svojstva programskih jezika?
\item Koji formalizam se koristi za opisivanje sintakse programskog jezika?
\item Šta definiše semantkika programskog jezika?
\item Koji su formalni okviri za definisanje semantike programskih jezika?
\item Šta je ime?
\item Šta je povezivanje?
\item Koja su moguća vremena povezivanja?
\item Šta je doseg?
\item Šta je kontrola toka?
\item Koji su mehanizmi odredivanja kontrole toka?
\item Šta je sistem tipova i šta on uključuje?
\item Šta je tipiziranje i kakvo tipiziranje postoji?
\item Kada se radi provera tipova?
\item Koja je razlika između kompiliranja i interpretiranja?
\item Šta je rantajm sistem?
\end{enumerate}

\end{multicols}
\end{document}